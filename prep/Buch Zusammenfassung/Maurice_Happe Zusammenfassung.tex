\documentclass{article}
\usepackage[utf8]{inputenc}
\usepackage{ngerman}
\title{Zusammenfassung: Texten für die Technik}
\author{Maurice Happe}
\date{1.November 2018}

\begin{document}

\maketitle

\section{Einleitung}
Da in den vergangenen Jahren Bachelor Informatik Studenten Probleme mit dem Verfassen von wissenschaftlichen Arbeiten hatte, wird ihnen das Buch \glqq Texten für die Technik\grqq{}  empfohlen. Es ist geschrieben von Andreas Baumert und Anette Verhein-Jarren. Es wurde 2016 vom Springer Verlag veröffentlicht.
Das Buch beschreibt die korrekte Herangehensweise beim Schreiben einer wissenschaftlichen Arbeit und deren konkrete Umsetzung.
In den folgenden Kapiteln ist das Buch zusammengefasst.
\section{Vorbereitung}
\subsection{Planung}
Damit eine wissenschaftliche Idee oder Erfindung an die Öffentlichkeit gelangt, ist das Schreiben einer wissenschaftlichen Arbeit essentiell.
Dazu gehört auch die Dokumentation über alle Ideen und Misserfolge, die zu dem Ziel geführt haben. Das Dokumentieren von Misserfolgen ist insofern wichtig, damit diese Fehler in der Zukunft gemieden werden können.
Ein Deckblatt mit allen Metadaten ist notwendig, um dem Leser einen groben Überblick zu verschaffen.

Zum Schreiben in Teams ist eine koordinierte Gliederung vorab zu Planen, um reibungsloses Arbeiten zu gewährleisten. Diese Gliederung kann diverse Strukturen haben, aber wichtig ist, dass sie anfangs eindeutig definiert ist.
Der Zeitplan muss flexibel zu ändern sein. Falls Arbeitsprozesse unterschätzt werden, muss die Zeiteinteilung neu überarbeitet werden.

Es gibt viele verschiedene Herangehensweisen. Wichtig ist, dass der Schreiber seinen eigenen Stil findet. Die drei Schritte \glqq Strukturierung, Schreiben und Überarbeitung\grqq{} können seriell als auch parallel ausgeführt werden.

\subsection{Informationsangabe}
In einem wissenschaftlichen Text muss genug Informationsgehalt sein, um zum Beispiel die Umstände und die Durchführung eines Experiments dem Leser zu beschreiben. Da ist es wichtiger eventuell zu viele Informationen zu haben, statt dass wichtige Informationen fehlen. Dazu gehört auch das Vorstellen von Related Work. 

\subsection{Zielgruppe}
Fragen mit der sich jeder Autor beschäftigen muss ist, welche Sprachkompetenz,welche Fachkenntnis und welches Interesse hat der durchschnittliche Leser des Artikels. In der Regel ist es hilfreich kurze aussagekräftige Sätze zu formulieren. Fachwörter benötigen gegebenenfalls eine einmalige Erklärung. Ein Software-Entwickler interessiert sich zum Beispiel eher für den Programm-Code, als ein Normalverbraucher einer Software. Hier muss abgewägt werden an wen sich der Text richtet. Es ist möglich, dass verschiedene Lesertypen sich auf verschiedene Bereiche der Arbeit fokussieren. So lässt sich jedes Kapitel für einen anderen Lesertypen anpassen.
Genauso wichtig ist es für den \glqq eiligen Leser\grqq{} alle nötigen Informationen schnell auffindbar zu machen.
\section{Das Schreiben}
\subsection{Appellebene ausformulieren}
Hinter jedem Satz befindet sich eine Appellebene. Doch diese ist für jeden Leser nicht immer eindeutig. Deswegen wird geraten den genauen Appell besser auszuformulieren. Je nach Intention muss die Wortwahl und Formulierung anders sein. Es gibt bestimmte Signalwörter die implizit dem Leser klar machen, was vom ihm erwartet wird.

\subsection{Korrekte Wortwahl}
Bei der Verwendung von Verben muss die korrekte Verbform benutzt werden.
Damit ist nicht der grammatikalische Zusammenhang gemeint, sondern der inhaltliche Kontext.
Dazu gehört das korrekte Tempus. Im Normalfall wird im Präsenz geschrieben. 
Die erste Person Singular wird immer gemieden und kann durch den Passiv umgangen werden. Bei Anweisungen an den Leser braucht man den Imperativ. Falsch platzierte Modalverben verwirren den Leser und deuten falsche Intentionen an.

Auch beim Substantiv muss auf einiges geachtet werden. Da auch Nicht-Muttersprachler deutsche Texte lesen wird geraten, dass eine simpler Kasus verwendet wird. Auch Wortkombinationen aus mehr als drei Wörter, sollten im Normalfall gemieden werden. Bindestriche zwischen Wortkombinationen helfen den Lesefluss.

Die Verwendung von Adjektiven sollte nur auf essentielle limitiert werden. Die meisten Adjektive lassen zu viel eigene Meinung einfließen, weswegen man diese meiden sollte.

Präpositionen helfen zur lokalen Beschreibung, aber sollten pro Satz nicht zu oft vorkommen, weil es den Leser verwirren kann.
Beim Verwenden von Abkürzungen von fachbezogenem Inhalt ist eine kurze Erklärung für den Leser hilfreich. Aber die Verwendung von zum Beispiel\glqq u.s.w. oder d.h.\grqq{} wird eher abgeraten. 
Ebenso sollten Fremdwörter ganz gemieden werden, um nichts falsches zu suggerieren. Dasselbe gilt für \glqq Füll- und Blähwörtern \grqq{}, da sie für das Verständnis nicht notwendig sind.

Mehrfache Verwendung von einem selben Wort, kann zu schlechter Lesbarkeit führen. Dies kann mit Synonymen ausgewichen werden.
Doch sollte man bei Fachwörtern nicht zu umgangssprachlichen Synonymen greifen.

\subsection{Satzbau}
Beim Schreiben, fällt auf, dass es leichter ist sich mündlich zu formulieren, weil es beim Schreiben viele Regeln und Vorschriften gibt, über die man im mündlichen Dialog hinweg sehen kann. 
Bei längeren Sätzen muss darauf geachtet werden, dass das Verb durch Nebensätze nicht zu sehr aufgeteilt ist.
Als Faustregel gilt: \glqq Eine Aussage pro Satz \grqq{} .
Wenn eine Aufzählung den Satz zu lang zieht, lässt sich eine Liste erstellen, die dasselbe übersichtlicher zusammenfasst.

Grundsätzlich ist ein kürzerer Satz immer besser als ein längerer Satz mit dem selben Inhalt, da er dasselbe kompakter aussagt. 
Wenn der Satz zu lang ist, aber die Informationsdichte bereits minimal ist, dann lässt sich der lange Satz auch in zwei oder drei kleinere Sätze ausformulieren.
Hier darf nicht verwechselt werden zwischen Füllwörtern und Wörter die inhaltlich notwendig sind, aber aus Fachkontext weggelassen werden können.
Zum Beispiel \glqq Es ist neun [Uhr]\grqq .
Es wird davon abgeraten, nominalisierte Verben zu verwenden, da es den Satzbau komplizierter macht.

Der Satzbau kann auch kompliziert gestaltet werden, wenn mit dem Objekt angefangen wird und dem Leser nicht direkt klar ist, dass es sich nicht um das Subjekt handelt.
Im Zweifelsfall ist die typische Reihenfolge Subjekt, Prädikat und Objekt immer am verständlichsten. 
Dennoch muss auch darauf geachtet werden, dass Pronomen auf das letzte Substantiv referenziert. Ist dies nicht der Fall muss man das eigentliche Nomen verwenden, weil es auch manchmal nicht aus dem Kontext klar ist.  

Klammern hilft zum besseren Verstehen, aber kann im Übermaß auch verwirren und den Lesefluss beeinträchtigen. Somit muss abgewägt werden, ob ein neuer seperater Satz doch besser wäre.

\subsection{Strukturierung}
Für die Strukturierung gibt es für bestimmte Texttypen konkrete Normen, an die man sich halten muss. Ansonsten bietet Related Work eine gute Orientierung zur Struktur. Wichtig ist, dass im abstract schon die wesentlichen Informationen gegeben sind. Zu Beginn eines Kapitels hilft ein Unterkapitel mit einer groben Zusammenfassung, dem Leser kurzgefasst das nötigste mitzugeben.

\section{Nachbearbeitung}
Ob auf sprachlicher Richtigkeit oder inhaltlicher Korrektheit muss ein Artikel überprüft werden. Es muss die Zurechenbarkeit der jeweiligen Abschnitte und deren Quellen stimmen. Auch die Grammatik, Rechtschreibung und schwache Formulierungen werden überarbeitet.
\section{Zusammenfassung}
Es gibt gewisse Richtlinien für das Schreiben einer Arbeit. Doch in einigen Fällen geht es darum eine goldene Mitte zu finden. 
Wichtig ist es regelmäßig die Texte nach den Kriterien zu korrigieren.
In jedem Fall ist es immer von Vorteil, wenn Testleser von der geschätzten Zielgruppe Rückmeldung geben und man auf der Kritik basierend den Text überarbeitet.
\begin{thebibliography}{1}
\bibitem[1]{1}
Texten für die Technik 2. Auflage, Andreas Baumert und Annette Verhein-Jarren, Springer Vieweg, 2016, ISBN: 978-3-662-47409-9
\end{thebibliography}
\end{document}
